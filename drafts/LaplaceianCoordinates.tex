\documentclass[12pt]{article}
\usepackage{amsmath,amssymb, amsthm}
\usepackage{graphicx}
\usepackage{times}

\usepackage{geometry}
 \geometry{
 a4paper,
 total={170mm,257mm},
 left=20mm,
 top=20mm,
 }

\title{Laplacian coordinates for a graph}
\begin{document} 
 \maketitle

 \section{Definitions}
 Let $G=(V,E)$ be a graph with edges $E$ and vertice $V$.

 Let $v_1,\ldots,v_k \in V$ be a set of vertices we call
 landmarks. Denote $B = \{v_1,\ldots,v_k\}$

 For $i = 1,\ldots,k$ define the diffusion function $f_i:V \to R$ as
 follows.
 \begin{itemize}
 \item  $f(v_i) = 1$
 \item For $1\leq j \leq k, j \neq i$, $f(v_j) = 0$
 \item $f_i(v)$ for $v \notin B$ satisfy the diffusion operator $L-I$
 \end{itemize}

 \section{Coordinate System}

 We define a coordinate system  $C:v \to (f_1(v),\ldots,f_k(v))$.

 We say that the coordinate system is complete if for any $u,v \ in
 V$, if $\|C(u) - C(v)\| \leq \epsilon$ then $d(u,v) \leq a
 \epsilon$ for some appropriately defined measure of distance $d$.

 One can use diffusion metric or maybe there is a way to consider
 adding $u$ or $v$ to the landmarks and checking whether the result is over-complete.

 \section{Claim}
 I believe that if the vertices are points on a  manifold of dimension
 $d$ then there is a set of $d+1$ landmarks that create a complete
 coordinate system.

 On the other hand I don't think that having doubling dimension $d$ is
 sufficient. For example, if the graph is disconnected then we need at
 least one landmark on each connected component. I also think that
 having one connected component + small doubling dimension is enough.
 
 
\end{document}
